\subsection{Javascript Runtime}
\label{subsec:jsruntime}
Mithilfe des Javascript Runtime Interfaces, lassen sich Javascript Methoden aus C\# Code
aufrufen. In dem Interface befindet sich eine Methode names \emph{InvokeAsync}, die jedoch in zwei
Varianten zur Verfügung steht:
\begin{itemize}
    \item ValueTask<TValue> InvokeAsync<TValue>(string, object?[]?)
    \item ValueTask<TValue> InvokeAsync<TValue>(string, CancellationToken, object?[]?)
\end{itemize}
Der zweiten Funktion kann ein \emph{CancellationToken} mitgegeben werden, um die Methode manuell
abzubrechen \cite{JsRuntime}[vgl.].
\newline
\newline
Des Weiteren gilt für die Verwendung des Javascript Runtime Interface Folgendes:
\begin{itemize}
    \item Der \emph{string} Parameter in InvokeAsync steht für den Javascript Methodennamen.
    \item Der \emph{object?[]?} Parameter ist ein Array von Objekten, um
    der Javascript Methode die Parameter zu übergeben.
    \item TValue ist ein Template-Objekt, welches als Rückgabewert benutzt werden kann.
    Zudem muss das Objekt mithilfe von \emph{JSON} serialisierbar sein.
    \item Der Javascript Code muss unter dem Verzeichnis \emph{wwwroot} vorhanden sein.
    \item Um diesem Mechanismus anzuwenden, muss in der Komponente das
    Interface \emph{IJSRuntime} \emph{injected} werden
    \item Bei Blazor Server können Javascript Methoden erst aufgerufen werden, sobald der
    Signal R Channel aufgebaut ist.
\end{itemize}

Um diesen Mechanismus zu demonstrieren, soll im Folgenden ein Beispiel dargestellt werden.
In diesem Beispiel wird ein String in einer Javascript-Methode erzeugt und zurückgegeben. Der
erzeugte String wird dann im C\# Code abgefangen und auf der Benutzeroberfläche
angezeigt.

\lstinputlisting[language=JavaScript,caption={Javascript-Methode generateString},
    label=lst:jsMethodeGenerateString]{\srcloc/Blazor/test.js}

\lstinputlisting[language={[Sharp]C},caption={Beispiel einer Javascript-Komponente},
    label=lst:jsExampleComponent]{\srcloc/Blazor/JsExample.razor}