\subsection{Vorgehen in der Datenwissenschaft}
\label{subsec:ds_vorgehen}
Ein Projekt welches in der Datenwissenschaft (engl. Data Science) angesiedelt ist, beginnt in der Regel mit einem geschäftlichen Problem, so wie es auch in dieser Thesis der Fall ist. Sobald das Problem klar definiert ist, sollten ein oder mehrere Konzepte ausgearbeitet werden, wie das Problem gelöst werden kann. Diese Konzepte sollen als Leitfaden dienen um das angestrebte Ziel zu erreichen. In der Regel ist es ratsam mehr als ein Konzept auszuarbeiten, da so ausweichmöglichkeiten festgelegt werden können, sollte ein Konzept nicht funktionieren. So werden auch in dieser Arbeit, in dem Kapitel \emph{\nameref{chap:konzepte}}, für das vorliegende Problem Konzepte ausgearbeitet und Evaluiert. 
\newline
\newline
Mit der klaren Definition des Problem und der darauffolgenden Konzeptionieren, kann der Datenwissenschaftler mit Hilfe des \emph{OSEMN}-Vorgangs die Problematik angehen \cite{Vorgehen_ds}. 
\newline
\newline
Der \emph{OSEMN}-Vorgang besteht aus den folgenden Elementen:

\begin{itemize}
    \item Obtian data (Erhalten von Daten)
    \item Scrub data (Daten reinigen)
    \item Explore data (Untersuchen von Daten)
    \item Model data (Modelldaten)
    \item Interpret results (Interpretieren von Ergebnissen)
\end{itemize}

\subsubsection{Obtian data}
Die wichtigste Ressource des 21. Jahrhundert besteht in den Daten, die zur Verfügung stehen. Dies erläuterte  Klaus Schwab, der Gründer des Weltwirtschaftsforums, mit seinen Worten: \emph{Die wertvollste Ressource des 21. Jahrhunderts sind nicht mehr Öl, sondern Daten}. Aufgrund dessen besteht der erste Schritt, nach der Evaluierung der Konzepte, in der Beschaffung der Daten. Es muss zunächst ein Überblick geschaffen werden. Zum Überblick gehören Informationen wie:

\begin{itemize}
    \item Welche Daten bereits vorhanden sind.
    \item Welche Daten eventuell noch intern neu erworben werden müssen.
    \item Welche Daten aus dem Internet gezogen werden können
\end{itemize}

Sobald die Daten beschaffen worden sind, kann mit dem nächsten Schritt fortgefahren werden.

\subsubsection{Scrub data}
Der nächste Schritt besteht darin, die Daten zu bereinigen. Zum bereinigen der Daten gehört der Vorgang mit dem die Daten in ein standardisiertes Format gebracht werden. Dazu gehört der Umgang mit fehlenden Daten, sowie die Korrektur von Fehlern und das Entfernen von sogenannten \emph{outlier} \cite{Vorgehen_ds}. Outlier sind Daten, welche im Verhältnis zu der gesamten Datenmenge aus der reihe Tanzen. 

\subsubsection{Explore data}
Die Datenuntersuchung oder auch Datenanalyse dient dazu, um mit den Daten vertraut zu werden und ein besseres Verständnis für die Daten zu Gewinnen. Dies ist ein sehr Wichtiger Schritt bei einem \emph{Data Science} Projekt, da nur dann ein gutes Ergebnis erzielt werden kann, wenn die Daten, mit den gearbeitet werden kann, verstanden sind. Das Verständnis über die Daten trägt zu dem auch maßgeblich dazu bei, den richtigen Ansatz für die Modellierung zu finden.

\subsubsection{Model}
Nach dem Erforschen der Daten, kann ein \emph{Machine Learning} Modell eingesetzt werden. Es existieren viele verschieden Modelle, die meist einen für einen speziellen Fall implementiert worden sind. Die Auswahl des Modells hängt ganz davon ab, welche Art von Problem gelöst werden soll und welche Daten vorhanden sind, um dieses Problem zu lösen. 

\subsubsection{Interpret results}
Der letzte Schritt besteht darin, die gesammelten Ergebnisse zu Interpretieren. Dazu gehörend ist die Entscheidung, ob die erzielten Ergebnisse gut oder schlecht sind. Meistens werden zur Hilfe der Entscheidung ob die Ergebnisse gut oder schlecht sind, Diagramme, Grafiken und Tabellen erstellt. Es soll hier zudem auch Entschieden werden, ob die Ergebnisse verwendet werden sollten oder ob nicht weiter geforscht werden muss. So entsteht ein Kreislauf, der mit dem erstellen weiteren Konzepte startet und mit dem Interpretieren der Ergebnisse endet. 
