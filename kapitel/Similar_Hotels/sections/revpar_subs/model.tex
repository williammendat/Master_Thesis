\section{Modellbildung}
\label{subsec:revpar_model}

Nachdem der vorbereitete Datensatz vorliegt, wird das Modell implementiert. Aufgrund des Vorhandenseins vieler kategorischer Daten wird erneut das CatBoost-Modell verwendet. Die Implementierung erfolgt ähnlich wie in der vorherigen Sektion zur Identifizierung ähnlicher Hotels.

\begin{lstlisting}[language=Python, label=lst:revpar_model, caption=Erzeugung der RevPAR-Vorhersagen]
import catboost as cb

cat_features = [
    "stay_date_weekDayName",
    "stay_date_monthName",
    "holiday",
    "time_to_arrival"
]
    
result_dict = dict()
    
for id in silimar_hotel_ids:
    X_train, y_train, X_test, y_test = get_data(id)
    model = cb.CatBoostRegressor()
    model.fit(X_train, y_train, cat_features=cat_features)
    y_pred = model.predict(X_test)
    result_dict[id] = y_pred
    
df = pd.DataFrame(result_dict)
overall_preds = df.astype(float).mean(axis=1)
\end{lstlisting}

Die Code-Zeilen in Listing \ref{lst:revpar_model} erzeugen die Variable \emph{overall\_preds}, die eine Liste von normierten RevPAR-Werten darstellt. Diese normierten RevPAR-Werte werden als Vorhersage für das Hotel ohne Vergangenheitsdaten verwendet.
    