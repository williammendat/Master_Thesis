\section{Evaluation}
\label{sec:eval_konzept}
Nachdem nun die ausgearbeiteten Konzepte vorgestellt wurden, gilt es diese zu Bewerten um festzulegen mit welchen Konzepten fortgefahren werden soll. Jedes der vorgestellten ist auf seine Art valide und hat auch Berechtigung verfolgt zu werden. Um deshalb entscheiden zu können, welches Konzept überhaupt nachgegangen werden soll oder in welcher Reihenfolge die Konzepte ausprobiert werden sollen, werden die Konzepte nach den folgenden Kriterien bewertet:
\begin{itemize}
    \item Aufwand
    \item Erfolgswahrscheinlichkeit
    \item Impact
\end{itemize}
Aufwand und Erfolgswahrscheinlichkeit sind selbsterklärend. Der Impact bezieht sich darauf, in wie fern happyhotel im generellen von dem Konzept profitieren könnte und ob das Konzept nicht auch für schon vorhandene Kunden eingesetzt werden könnte.
\newline
\newline
Jedes Konzept kann bei jedem Kriterium eine Zahl zwischen 1 bis 5 erzielen, wobei 5 das Beste und 1 das Schlechteste in dem jeweiligen Kriterium bedeutet. Die Ergebnisse der Evaluation sind wie folgt in der Tabelle dargestellt:
\begin{table}[ht]
    \centering
    \begin{tabularx}{\textwidth}{|d|c|c|c|c|}
        \textbf{Konzepte} & \textbf{Aufwand} & \textbf{Erfolgsw.} & \textbf{Impact} & \textbf{Result} \\
        \hline
        Daten von vielen Hotels & 4       & 4            & 4      & 12 
        \\\rowcolor{Gray}
        Mitbewerber             & 4       & 3            & 3      & 10                \\
        Ähnliche Hotels         & 4       & 5            & 4      & 13  
        \\\rowcolor{Gray}
        Synthetischen Daten     & 1       & 3            & 5      & 9   \\
    \end{tabularx}
    \caption[Evaluierung der Konzepte]{Evaluierung der Konzepte}
    \label{table:eval_kozepts}
\end{table}

Nach der Evaluierung wurde bestimmt, dass das Konzept \emph{\nameref{sec:aehnliche_hotels}} das größte Potenzial hat und soll dementsprechend auch verfolgt werden. Je nach Zeit und Ergebnisse werden die Konzepte \emph{\nameref{sec:Mitbewerber}} und \emph{\nameref{sec:all_Hotels}} auch verfolgt und evaluiert werden. Die Erstellung einer Simulation durch Synthetische Daten ist auch ein sehr interessantes Konzept, würde aber im rahmen dieser Thesis zu weit gehen.