\subsection{Momentane Problematik}
\label{subsec:problematik}
Wie es Albert Einstein schon sagte: Soll die Zukunft vorhersagen gesagt werden, so sollte sich die Vergangenheit angeschaut werden. Auf diesem Grundprinzip ist happyhotel auch vorgegangen, sie schauen sich bei beiden Ansätzen die Vergangenheit an um Vorhersagen über die Zukunft zu tätigen. 
\newline
\newline
Doch was passiert, wenn die Daten der Vergangenheit nicht vorhanden sind? Dies kann zum Beispiel passieren wenn ein Hotel die Software nutzen möchte, welches erst in der Zukunft eröffnet. Auch dieses Hotel soll mit adäquaten Preisvorschlägen gefüttert werden.
Die soeben beschriebene Situation ist ein generelles Problem im Machine Learning Bereich. Es kommt nicht allzu selten vor, dass keine Vergangenheitsdaten aus den verschiedensten Gründen vorliegen. Dieser Problematik soll in dieser Arbeit auf dem Grund gegangen werden. 
\newline
\newline
Ziel dieser Arbeit ist es,  ein Modell zu entwickeln welches Preisempfehlungen für ein Hotel liefert, für das es bisher noch keine Vergangenheitsdaten gibt. Dieses Modell soll dann für folgende zwei Scenarien genutzt werden können:
\begin{itemize}
    \item Neue happyhotel Kunden ohne Daten
    \item Nachfrageeinschätzung für bestimmte Märkte
\end{itemize}