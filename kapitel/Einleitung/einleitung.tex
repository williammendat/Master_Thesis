\chapter{Einleitung}
\label{chap:einleitung}
In einer Welt, die sich mit rasanter Geschwindigkeit digitalisiert, suchen die Menschen stehts nach Wegen, um die Komplexität des modernen Lebens zu bewältigen. Diese Digitalisierung hat eine stetig wachsende Sehnsucht nach der Vorhersage zukünftiger Ereignisse hervorgebracht - sei es in der Wirtschaft, der Gesundheitsbranche oder auch im Bereich des Dienstleistungssektors wie dem Hotelgewerbe. Es ist ein Streben nach Präzision, ein Bestreben, aus Daten und Mustern eine art Kristallkugel zu formen, um die Zukunft vorhersagen zu können.
\newline
\newline
Albert Einstein hat einst mit einem Buchtitel von Ihm gesagt: 
\begin{zitat}
    If you want to know the future, look at the past. \cite{AE_zitat}
\end{zitat}
Dieser Gedanke illustriert die gängige Annahme, dass die Vergangenheit Hinweise auf die Zukunft liefern kann. Es ist interessant anzumerken, dass dieses Zitat auch als Titel eines Buches von Einstein dient, welches seine philosophischen Ansichten zur Zeit, Raum und Vorhersage behandelt.
\newline
\newline
Doch was passiert, wenn diese Vergangenheitsdaten nicht verfügbar sind oder nicht genutzt werden können? In Branchen wie der Hotelindustrie, die oft noch auf traditionelle, statische Preisstrategien zurückgreifen, stellt sich die Frage, wie eine effektive Vorhersage ohne spezifische historische Daten möglich ist. 
\newline
\newline
Es wird immer deutlicher, dass ein dynamischerer Ansatz im Hotelwesen erforderlich ist, um die Umsatzoptimierung durch Revenue Management zu steigern. Dies erfordert die Anpassung von Preismodellen an sich ändernde Nachfrage und andere Einflussfaktoren. Eine mögliche Lösung liegt in der Verlagerung der traditionellen Rolle des Revenue Managements auf Modelle, die auf breiteren Datenquellen und fortgeschrittenen Methoden des maschinellen Lernens basieren.
\newline 
\newline
Die Suche nach einem solchen Modell, das ohne die spezifischen Vergangenheitsdaten eines bestimmten Hotels auskommt, bildet das Herzstück dieser Forschungsarbeit. Der Fokus liegt darauf, alternative Datenquellen zu erkunden und innovative Ansätze zu entwickeln, um Prognosen und Entscheidungsgrundlagen für das Revenue Management in der Hotellerie zu schaffen. Ziel ist es, dass diese nicht ausschließlich auf vergangenen Daten eines spezifischen Hotels basieren, sondern auf einer Vielzahl von allgemeinen, zugänglichen Informationen und fortschrittlichen Analysemethoden beruhen. Es geht darum, einen Weg zu finden, wie Hotels, selbst ohne ihre spezifischen vergangenen Daten, zukünftige Entscheidungen im Bereich des Revenue Managements treffen können, um ihre Leistung zu optimieren und ihre Wettbewerbsfähigkeit zu stärken.