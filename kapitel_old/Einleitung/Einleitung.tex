\chapter{Einleitung}
\label{chap:einleitung}
Die Menschheit ist heutzutage darauf fokussiert, die komplette Welt zu digitalisieren. Dabei
existiert ein Grundsatz:
\begin{zitat}
    Alles, was digitalisiert werden kann, wird digitalisiert \cite{digitalisierteWelt}
\end{zitat}

Um dies zu realisieren ist es vonnöten überall Hardware und Software zu verbinden. Sei es nun das
Handy, mit welchem durch nur einen klick die Bankdaten angezeigt werden können, oder ein
selbstfahrendes Auto, welches einen Anwender selbstständig zum Ziel fährt. Dies sind nur zwei
Beispiele einer unendlich langen Liste. Hinter diesen technischen Wundern stecken meist
mehrere Tausend kleiner Mikrocomputern und Mikrocontrollern, die dann mittels Software zusammen
interagieren. Die Kombination dieser zwei Komponenten werden durch den Oberbegriff „Embedded
System“ oder auch zu Deutsch ein „Eingebettetes System“ zusammengefasst.
\newline
\newline
Embedded Systems können dabei grundsätzlich zwischen zwei Plattformen unterschieden werden:
\begin{itemize}
    \item Deeply Embedded System
    \item Non-Deeply Embedded System
\end{itemize}
\newline
\newline
Deeply Embedded Systems sind die wesentlichen Bausteine des Internet of Things
\cite{HochschuleniederrheimDeeply}. Die Anwendungen,
die bei Deeply Embedded Systems implementiert werden, basiert auf speziell angepassten
Echtzeitbetriebssystemen. Zusätzlich werden die Programmiersprachen C und C++ sowie ganz spezielle
\ac{gui}-Frameworks wie zum Beispiel TouchGFX verwendet.
\newline
\newline
Anders als bei den Deeply Embedded Systems, die sehr auf speziellen Technologien aufbauen, bieten
Non-Deeply Embedded Systems eine höhere Flexibilität in Sachen Technologien an. Von Seiten der
Programmierung ist es möglich, unterschiedliche Technologien und Programmiersprachen zu verwenden.
Dort gilt bis dato unter Linux die Kombination von C++ und Qt für \ac{gui}-lastige Systeme als
„State of the Art“.
\newline
\newline
Die Kombination aus C++ und Qt funktionierte bislang sehr gut. Jedoch kommt dieser Ansatz
auch mit Herausforderungen. Nicht nur die höheren Entwicklungszeiten für die
Entwicklung von C++ Anwendungen, sondern auch die geringe Verfügbarkeit von Experten auf dem
Arbeitsmarkt sorgen für schlechtere Softwarequalität und längere Produktionszeiten.
\newline
\newline
Um diesen Herausforderungen zu entgegnen, soll in dieser Abschlussarbeit ein anderer Ansatz
betrachtet
werden. Und zwar könnte sowohl die Anwendungsschicht als auch die Persistenzschicht als .Net
Core Anwendungen implementiert werden. Als \ac{gui}-Technologie soll dabei die neue
Microsoft-Technologie namens \emph{Blazor} als Qt-Ersatz zum Einsatz kommen. Somit kann erreicht
werden, dass die komplette Applikation in .Net Core implementiert werden kann.
\section{Aufgabenstellung}
\label{sec:aufgabenstellung}
Das Ziel dieser Arbeit ist die Entwicklung eines Blazor-basierten Frontends auf einem Raspberry Pi
4B. Dabei soll ein aussagekräftiger Vergleich zwischen den Technologien geschaffen werden, um
eine mögliche Verdrängung mittels Blazor zu demonstrieren.
Dazu soll zunächst veranschaulicht werden, wie der momentane Stand der Technik für Non-Deeply
Embedded Systems aussieht. Anschließend soll die Codebasis auf .Net Core und Blazor
gewechselt werden.
Insbesondere sollen dabei verschiedene Aspekte, wie zum Beispiel das Verhalten zur Laufzeit,
dieses Ansatzes überprüft werden.
\begin{figure}[h]
    \centering
    \includegraphics[width=0.6\textwidth, center]{Einleitung/blazorxraspberry}
    \caption[Blazor mit Raspberry Pi]{Blazor mit Raspberry Pi}
    \label{img:blazorxraspberry}
\end{figure}
\section{Verwendete Hardware}
\label{sec:verwendeteHardware}
Als Zielplattform für diese Arbeit dient ein Raspberry Pi 4B. Der Raspberry Pi 4B
verfügt dabei unter anderem über die folgenden technischen Spezifikationen: \cite{RasberryPiSpecs}
\begin{itemize}
    \item 1,5 GHz ARM Cortex-A72 Quad-Core-CPU
    \item 4 GB LPDDR4 SDRAM
    \item Gigabit LAN RJ45 (bis zu 1000 Mbit)
    \item Bluetooth 5.0
    \item 2x USB 2.0 / 2x USB 3.0
    \item 2x microHDMI
    \item 5V/3A @ USB Typ-C
    \item 40 GPIO Pins
    \item Mikro SD-Karten Slot
\end{itemize}
Der Raspberry Pi 4B wird mit dem \emph{Raspbian Buster with desktop} auf der SD-Karte betrieben.
\newline
Um noch mehr Funktionalität aufbringen zu können, wurde auf dem Raspberry Pi ein
\emph{RPI SENSE HAT Shield} aufgesteckt. Mithilfe des \emph{RPI SENSE HAT Shield} können dann
unter anderem Daten, wie zum Beispiel die momentane Temperatur, oder auch die momentane
Luftfeuchtigkeit gewonnen werden. Zudem ist auf dem Raspberry Pi noch eine 8x8 LED-Matrix enthalten
und ein Joystick mit 5 Knöpfen, die eingelesen werden können.
\begin{figure}[h]
    \centering
    \includegraphics[width=0.5\textwidth, center]{Einleitung/pi-sense}
    \caption[Raspberry Pi 4 B]{Verwendeter Raspberry Pi}
    \label{img:piSense}
\end{figure}


\section{Verwendete Software}
\label{sec:verwendeteSoftware}
Im Rahmen dieser Arbeit werden die folgenden Softwaretools zur Entwicklung eingesetzt:
\begin{itemize}
    \item Um eine Visualisierung des Images \emph{Raspbian Buster with desktop} von dem Raspberry
    Pi zu erhalten, wird die Windows Desktop Anwendung \emph{VNC Viewer} verwendet. Dadurch ist
    die Möglichkeit gegeben, bequem und einfach den Raspberry Pi Remote zu bedienen.
    \item Für die Demonstrierung des Kapitels \emph{\nameref{chap:standTechnik}} wird eine
    beispielhafte grafische Oberfläche mithilfe des Qt-Creators implementiert.
    \item Den Zugriff auf die Funktionalitäten des \emph{RPI SENSE HAT Shield} wird mittels der
    \emph{RTIMUlib} Bibliothek realisiert.
    \item Die Programmiersprachen, die Hauptsächlich in dieser Arbeit verwendet werden, sind C++
    und C\# beziehungsweise .Net Core.
    \item Um den Zugriff auf die Funktionalitäten des \emph{RPI SENSE HAT Shield} mittels .Net
    Core zu gewährleisten, wird die IoT Bibliothek von Microsoft verwendet.
    \item Die Implementierung der Anwendung mit Blazor wird mittels der
    kostenlosen IDE \emph{Visual Studio Code} realisiert.
    \item Um zwischen dem Host Rechner und dem Raspberry Pi Dokumente und Ordner auszutauschen,
    wird die Desktop-Applikation \emph{WinSCP} verwendet.
\end{itemize}

Die oben vorgestellten Tools und Programmiersprachen wurden nicht explizit vorgegeben.
Dementsprechend sind selbst ausgesucht und sind zu Beginn dieser Arbeit schon alle
komplett eingerichtet.
