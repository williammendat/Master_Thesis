\subsection{Signal und Slot Konzept}
\label{subsec:signalslot}
Eine interaktive Benutzeroberfläche muss auf
Ereignisse reagieren können. Beispielsweise ist das Betätigen eines Buttons ein Auslöser für ein
Ereignis auf welches reagiert werden kann. Es gibt viele Methoden und Muster in welche ein
solches Ereignis-Aktionskonzept implementiert werden kann.

Ein \emph{Signal} ist im einfachen Sinne eine Nachricht, die versendet werden kann. Die Nachricht
signalisiert, dass sich der momentane Status eines Objektes geändert hat. Dahingegen ist ein
\emph{Slot} eine spezielle Funktion von einem Objekt, welche dann aufgerufen wird, wenn ein
bestimmtes \emph{Signal} gesendet wird \cite{GettingStartedQt}[vgl.].

Damit jeder \emph{Slot} auch weiß, auf welches \emph{Signal} reagiert werden soll, müssen diese verbunden werden. In der Sektion \emph{\nameref{subsec:programmierbeispiel}} war
folgende Zeile zu sehen:
\newpage
\begin{lstlisting}[language=C++, caption=Signal- und Slot-Beispiel, label=lst:SignalSlotBeispiel]
// Connect button with the App
QObject::connect(btnExit, SIGNAL(clicked()), &app, SLOT(quit()));

\end{lstlisting}

In dieser Codezeile fand die Verbindung zwischen einem Signal und einem Slot statt.
Wenn der Button betätigt wird, wird \emph{app} signalisiert und ruft \emph{quit()} auf. Die
Funktion \emph{quit()} beendet das Programm.

Einer der größten Vorteile dieser Methode im Gegensatz zu anderen Methoden ist, dass dadurch
n:m-Beziehungen abgebildet werden können. Das bedeutet also, dass sich ein Signal mit
beliebig viel Slots verbinden kann und dass sich ein Slot auf mehrere Signale verbinden kann
\cite{SignalSlotMechanismus}[vgl.].