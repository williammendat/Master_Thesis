\section{Installation}
\label{sec:installation}
Da in der vorherigen Sektion \emph{\nameref{sec:entwicklungsumgebung}} Visual Studio Code
eingerichtet wurde und mit dem Raspberry Pi
Remote verbunden wurde, kann das integrierte Terminal von Visual Studio Code für die
Installation von .Net Core verwendet werden.
\newline
\newline
Im ersten Schritt muss überprüft werden, ob es sich bei dem Raspberry Pi um die 32-bit oder die
64-bit
Version handelt. Dies kann mit dem Befehl \emph{uname -a} überprüft werden. Dabei können folgende
zwei Ergebnisse erfolgen:
\begin{zitat}
    Linux raspberrypi 4.19.97-v7l+ #1294 SMP Thu Jan 30 13:23:13 GMT 2020 armv7l GNU/Linux
    \newline
    Linux raspberrypi 5.10.60-v8+ #1291 SMP Thu Jan 30 13:21:14 GMT 2020 aarch64 GNU/Linux
\end{zitat}
Beim ersten Ergebnis handelt es sich um 32-Bit Version und beim zweiten um die 64-Bit Version.
\newline
\newline
Die jeweilige SDK kann entweder auf der offiziellen Microsoftseite oder in einem Terminal mit dem
Befehl \emph{wget} heruntergeladen werden. Nachdem der Download
beendet ist, kann mit den folgenden Befehlen die Installation beendet werden:
\begin{zitat}
    mkdir -p \$HOME/dotnet
    \newline
    tar zxf dotnet-sdk-6.0.100-rc.1.21458.32-linux-arm.tar.gz -C \$HOME/dotnet
    \newline
    export DOTNET\_ROOT=\$HOME/dotnet
    \newline
    export PATH=\$PATH: \$HOME/dotnet
\end{zitat}

Diese Befehle erstellen einen neuen Ordner namens \emph{dotnet}, entpacken die SDK und packen den
Inhalt in den neu erstellten Ordner. Zudem wird die \emph{Path-Variable} zu dem dotnet Ordner.
\newline
\newline
Mit dem Befehl \emph{dotnet --info} kann anschließend überprüft werden, ob die Installation
erfolgreich war.