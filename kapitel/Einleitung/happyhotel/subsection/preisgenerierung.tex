\subsection{Dynamische Preisgenerierung}
\label{subsec:Preisgenerierung}
Wie im vorherigen Kapitel erwähnt fokussiert sich happyhotel auf die Dynamische Preisgenerierung. Um diese Preise zu generieren braucht es vor allem zwei Sachen:
\begin{itemize}
    \item Die Daten des Hotels wie zum Beispiel Buchungen
    \item Ein Vorgehen um aus den gesammelten Daten Preise zu generieren
\end{itemize}

Die Daten bekommen sie aus den verschiedensten Quellen. Sogennante Property Management Systeme kurz PMS sind Systeme um ein Hotel zu verwalten. In diesem Property Management Systeme können Hotels zum Beispiel ihre Zimmer verwalten oder aber auch Buchungen anlegen und pflegen. Mit den Herstellern dieser Property Management Systeme arbeitet happyhotel zusammen um an die Daten des Hotels zu gelangen. 
\newline
Da die Daten vorhanden sind, braucht es ein Vorgehen um aus den Daten einen Preis zu generieren. Dazu ist happyhotel auf die folgenden zwei Ideen gekommen:
\begin{itemize}
    \item Buchungskurvenmodell
    \item Kombination aus RevPAR und Buchungskurvenmodell
\end{itemize}

Das Buchungskurvenmodell war die erste Idee von happyhotel. Bei dem Buchungskurvenmodell wird sich die Vergangenheit angeschaut um das zukünftige Buchungsverhalten vorherzusagen. Ziel dabei ist es die Auslastung für einen Tag vorherzusagen um anhand dessen einen akkuraten Preis zu bestimmen. 
\newline
\newline
Da bei dem Buchungskurvenmodell sich einige schwächen aufgezeigt haben, wurde ein neues Modell erschaffen um die Schwächen entgegen zu wirken. Es sollte nun der RevPAR wert vorhergesagt werden und mit dem Buchungskurvenmodell angepasst werden. RevPAR steht dabei für Revenue per Available Room. Auch bei diesem Modell wird sich die Vergangenheit des Hotels angeschaut um den Zukünftigen Umsatz pro verfügbarem Zimmer vorherzusagen und basierend darauf den endgültigen Preis zu ermitteln.