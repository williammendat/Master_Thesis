\subsection{Javascript Invokable}
\label{subsec:jsInvokable}
Nun kann es den Fall geben, dass C\# Code von Javascript aufgerufen werden muss. Dies wird mit
dem \emph{JSInvokable}-Attribut möglich. Die Funktion, die mit dem \emph{JSInvokable}-Attribut
erweitert wurde, muss zudem mit dem \emph{static-keyword} versehen werden.
\newline
\newline
Javascript-seitig werden von Blazor die Methoden
\begin{itemize}
    \item DotNet.invokeMethod (string, string, object[])
    \item DotNet.invokeMethodAsync (string, string, object[])
\end{itemize}
bereitgestellt, mit denen eine statische C\#-Methode aufgerufen werden kann. Dabei geben
Entwickler*innen mithilfe des ersten Parameters, den Project Namen an, in der sich die statische
Methode befindet. Der zweite Parameter, ist für den Namen der aufzurufenden Methode.
Mit dem dritten Parameter können weitere optionale Argumente übergeben werden.
\newline
\newline
Im Folgenden wird eine Komponente gezeigt die eine Javascript-Methode aufruft. In der
Javascript-Methode wird wiederum eine C\#-Methode aufgerufen. Das Ergebniss der C\#-Methode
wird anschließen in einem \emph{alert} angezeigt.
\newpage
\lstinputlisting[language={[Sharp]C},caption={Javascript Invokable Beispiel},
    label=lst:jsInvokableExample]{\srcloc/Blazor/JsInvokable.razor}

\lstinputlisting[language=JavaScript,caption={Javascript Invokable showGeneratedMessage},
    label=lst:jsInvokable]{\srcloc/Blazor/testInvokable.js}