% -------------------------------------------------------
% Daten für die Arbeit
% Wenn hier alles korrekt eingetragen wurde, wird das Titelblatt
% automatisch generiert. D.h. die Datei titelblatt.tex muss nicht mehr
% angepasst werden.

\newcommand{\hsmasprache}{de} % de oder en für Deutsch oder Englisch
% Für korrekt sortierte Literatureinträge, noch preambel.tex anpassen
% und zwar bei \usepackage[main=ngerman, english]{babel},
% \usepackage[pagebackref=false,german]{hyperref}
% und \usepackage[autostyle=true,german=quotes]{csquotes}

% Titel der Arbeit auf Deutsch
\newcommand{\hsmatitelde}{Entwicklung eines Machine Learning Modells ohne die Verwendung von Hotel spezifischen Vergangenheitsdaten}

% Titel der Arbeit auf Englisch
\newcommand{\hsmatitelen}{Development of a machine learning model without using hotel-specific historical data}

% Weitere Informationen zur Arbeit
\newcommand{\hsmaort}{Offenburg}    % Ort
\newcommand{\hsmaautorvname}{William} % Vorname(n)
\newcommand{\hsmaautornname}{Mendat} % Nachname(n)
\newcommand{\hsmadatum}{30.03.2024} % Datum der Abgabe
\newcommand{\hsmajahr}{2024} % Jahr der Abgabe
\newcommand{\hsmafirma}{happyhotel} % Firma bei der die Arbeit durchgeführt wurde
\newcommand{\hsmabetreuer}{Prof. Dr.-Ing. Janis Keuper, Hochschule Offenburg} % Betreuer an der
\newcommand{\hsmazweitkorrektor}{Prof. Dr. rer. nat. Klaus Dorer, Hochschule Offenburg} % Betreuer im
\newcommand{\hsmafakultaet}{EMI} % Fakultät
\newcommand{\hsmastudiengang}{INFM} % Studiengangsabkürzung.
% Diese wird in titelblatt.tex definiert. Bisher AI, EI, MK und INFM. Bitte ergänzen.

% Zustimmung zur Veröffentlichung
\setboolean{hsmapublizieren}{false}   % Einer Veröffentlichung wird zugestimmt
\setboolean{hsmasperrvermerk}{true} % Die Arbeit hat keinen Sperrvermerk

% -------------------------------------------------------
% Abstract

% Kurze (maximal halbseitige) Beschreibung, worum es in der Arbeit geht auf Deutsch
\newcommand{\hsmaabstractde}{In dieser Thesis wird ein Machine-Learning-Modell entwickelt, das gezielt darauf ausgelegt ist, Preisvorschläge ohne spezifische historische Daten zu generieren. Es werden verschiedene Ansätze vorgestellt, wobei einer davon detailliert untersucht wird. Anschließend erfolgt eine umfassende Evaluierung dieses Ansatzes, um die Performance zu überprüfen.

    Bei dem ausgearbeitetem Konzept wird versucht zu einem Hotel ähnliche Hotels zu finden um basierend auf den Hotels die Preisvorschläge zu generieren.

    Die Arbeit bietet einen Einblick in die Problematik der Datenwissenschaft, wenn keine historischen Daten verfügbar sind, und zeigt auf, wie dieser Herausforderung begegnet werden kann.}

% Kurze (maximal halbseitige) Beschreibung, worum es in der Arbeit geht auf Englisch

\newcommand{\hsmaabstracten}{In this thesis, a machine learning model is developed with the specific aim of generating price proposals without relying on specific historical data. Various approaches are presented, with one of them being examined in detail. Subsequently, a comprehensive evaluation of this approach is conducted to assess its performance.

    The elaborated concept involves attempting to find hotels similar to a given hotel in order to generate price proposals based on these similar hotels.

    This work provides insight into the challenge of data science when historical data is unavailable and illustrates how this challenge can be addressed.}
