\section{Embedded Systems}
\label{sec:EmbeddedSystems}
Ein Embedded System oder auf Deutsch ein eingebettetes System wird als eine integrierte,
mikroelektronische Steuerung angesehen. Welches meist darauf ausgelegt ist eine spezifische Aufgabe
zu erledigen \cite{EmbeddedSystem}[vgl.]. Dabei setzt sich ein Embedded System aus dem
Zusammenspiel zwischen
Hardware und Software zusammen. Solche Embedded Systems haben meist kein ausgeprägtes
Benutzerinterface und können weitergehend in die zwei Unterklassen, Non-Deeply und Deeply Embedded
Systems unterteilt werden.
\newline
\newline
Neben der logischen Korrektheit, die Embedded Systeme erfüllen müssen, lassen sie sich
durch eine Reihe unterschiedlicher Anforderungen und Eigenschaften von den heutzutage üblichen
Anwedungen abgrenzen. Unter anderem werden bei Embedded Systems ein sogenanntes
\emph{Instant on} gefordert, welches besagt, dass das Gerät unmittelbar nach dem Einschalten
betriebsbereit sein muss \cite{EmbeddedLinuxQuade}[vgl.].
\newline
\newline
Folgende Tabelle zeigt die typischen Anforderungen an ein Embedded System auf:

\begin{table}[ht]
    \centering
    \begin{tabularx}{\textwidth}{md}
        \textbf{Anforderung} & \textbf{Beschreibung}                                           \\
        \hline
        Funktionalität   & Die Software muss schnell und korrekt sein
        \\\rowcolor{Gray}
        Preis           & Die Hardware darf nicht zu kostspielig sein                 \\
        Robustheit       & Muss auch in einem rauen Umfeld funktionieren
        \\\rowcolor{Gray}
        Fast poweroff     & Muss in der Lage sein schnell das komplette System abzuschalten   \\
        Räumliche Ausmaße  & Muss klein sein, um sich in ein System einbinden zu können
        \\\rowcolor{Gray}
        Nonstop-Betrieb   & Muss in der Lage sein, im Dauerbetrieb laufen zu können          \\
        Lange Lebensdauer  & Muss in der Lage sein, teilweise mehr als 30 Jahre zu laufen
    \end{tabularx}
    \caption[Anforderungen an Embedded Systems]{Anforderungen an Embedded Systems
    \cite{EmbeddedLinuxQuade}}
    \label{table:AnforderungenEingebetteteSysteme}
\end{table}

\subsection{Hardware}
\label{subsec:EmbeddedHardware}
Die einzelnen Komponenten, die in Embedded Systems verbaut worden sind, entscheiden über
die vorhandene Leistung, den Stromverbrauch und die Robustheit, die das System ausmachen. Die
Kernkomponente eines Embedded System wird durch einen Prozessor repräsentiert und
wird häufig als \emph{System on Chip} eingebaut. Dabei ist die Tendenz zu den ARM-Core Modellen
steigend. Neben dem Prozessor befinden sich typischerweise auf den Embedded Systems
weitere Komponenten, wie zum Beispiel dem Hauptspeicher, dem persistenten Speicher und diversen
Schnittstellen, um weitere Peripheriegeräte anzuschließen \cite{EmbeddeHardware}[vgl.].
\newline
\newline
Damit weitere Peripheriegeräte angeschlossen werden können, müssen mittels einigen Leitungen
digitale Signale übertragen werden. Aufgrund dessen, dass Leitungen typischerweise nur über eine
begrenzte Leistung verfügen, werden Treiber eingesetzt, um peripherie Geräte zu verbinden.
Nicht selten kommt es vor, dass in einem Embedded System darüber hinaus noch eine galvanische
Entkopplung eingebaut wird. Dies dient dazu, die Hardware
vor Störungen von außen, wie Beispiel Motoren, zu schützen \cite{EmbeddedLinuxQuade}[vgl.].
\subsection{Software}
\label{subsec:EmbeddedSoftware}
Genauso wie sich Embedded Systems in zwei Bereiche unterscheiden lassen können, kann die
Software eines Embedded System in Systemsoftware und funktionsbestimmende
Anwendungssoftware unterteilt werden.
Für ein \emph{Deeply Embedded System} wird in den meisten fällen ein Echtzeitbetriebssystem
(Realtime Operating System - RTOS) verwendet, welches an die Hardware angepasst ist.
\newline
\newline
Ganz anders sieht es im \emph{Non-Deeply Embedded Systems} Bereich aus. Da diese für sehr
komplexe Aufgaben zum Einsatz kommen, kommt es nicht selten
vor, dass eine \ac{gui} für ein solches System vonnöten ist. Deswegen basiert ein
\emph{Non-Deeply
Embedded Systems} auf einer Systemsoftware, die Programmierer*innen mehr Möglichkeiten bei
der Entwicklung geben. Unter anderem ist die Möglichkeit gegeben, die Programmiersprache
und
das \ac{gui}-Framework selbst auszusuchen \cite{EmbeddedLinuxQuade}[vgl.].
