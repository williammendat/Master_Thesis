\subsection{Beschaffung von Region, City und Stadtgröße}
\label{subsubsec:region_city_size}
Durch eine schon im Vorfeld getätigte Arbeit, existieren zu jedem Hotel in unserer Datenbank die Koordinaten, repräsentiert durch die zwei Werte Long- und Latitude. Mithilfe von diesen zwei Werten sollte ein Skript geschrieben werden, um die Region, die Stadt und Größe der Stadt zu beschaffen. 
\newline
\newline
Für das Skript wurde \emph{Nominatim} API verwendet, um die einzelnen Werte zu beschaffen. Im Folgenden wird das Skript präsentiert:

\begin{lstlisting}[language=Python, label=lst:RS_Demo, caption=Einfaches Recommendation System für Film vorschläge]
    from geopy.geocoders import Nominatim

def get_region_city_size(latitude, longitude):
    geolocator = Nominatim(user_agent="city_size_app")
    location = geolocator.reverse((latitude, longitude), language='de')

    address = location.raw['address']

    if 'city' in address:
      region = address["city"]
      if 'state' in address:
        region = address["state"]
      return {
            "city": address["city"],
            "region": region,
            "size": "Großstadt"
        }
    elif 'town' in address:
        return {
            "city": address["town"],
            "region": address["state"],
            "size": "Kleinstadt"
        }
    elif 'village' in address:
      region = None
      if 'county' in address:
        region = address["county"]
      if 'state' in address:
        region = address["state"]
      return {
            "city": address["village"],
            "region": region,
            "size": "Kleinstadt"
        }
    else:
        return dict()
\end{lstlisting}

Die Funktion \emph{get\_region\_city\_size} nimmt als Parameter die Long- und Latitude-Werte und erzeugt dadurch ein \emph{Dictionary} mit den Werten \emph{region}, \emph{city} und \emph{size}.