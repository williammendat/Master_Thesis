\section{Ähnliche Hotels}
\label{sec:aehnliche_hotels}
Dieses Konzept der \emph{Ähnlichen Hotels} verschmilzt in gewisser Weise die Ideen der beiden Konzepte \emph{Mitbewerber Modell} und \emph{Hotel Daten von vielen Hotels}. Dieser Ansatz zielt darauf ab, ähnliche Hotels zu identifizieren und basierend auf den Daten dieser Hotels ein Modell zu entwickeln.
\newline
\newline
Im Gegensatz zum Konzept \emph{Hotel Daten von vielen Hotels} besteht bei diesem Ansatz die Möglichkeit, konkrete Buchungsdaten der jeweiligen Hotels zu verwenden. Die primäre Herausforderung liegt jedoch darin, die ähnlichsten Hotels zu identifizieren. Nachdem diese ähnlichen Hotels ausfindig gemacht wurden, kann das bereits vorhandene Modell \emph{Kombination aus RevPAR und Buchungskurve} ohne jegliche Anpassungen genutzt werden.
\newline
\newline
Dieses Modell wird dann, ähnlich wie bei anderen Hotels, mit den Buchungsdaten der identifizierten ähnlichen Hotels gefüttert, um Preise zu generieren. In diesem Szenario muss der Kunde lediglich eine Zuordnung zwischen dem RevPAR-Wert und dem konkreten Preis des Hotels festlegen.
\newline
\newline
Dieser Ansatz kombiniert die Vorteile beider vorherigen Konzepte, indem er sowohl auf die Ähnlichkeits-findung zwischen Hotels als auch auf die Nutzung spezifischer Buchungsdaten abzielt. Durch die Verwendung vorhandener Modelle ohne umfangreiche Modifikationen können so gezielt Preisvorhersagen für ähnliche Hotels generiert werden.