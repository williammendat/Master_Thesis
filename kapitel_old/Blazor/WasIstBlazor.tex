\section{Was ist Blazor?}
\label{subsec:wasIstBlazor}
Blazor ist ein Framework von Microsoft zum Erzeugen von Webseiten. Dabei macht Blazor gebrauch
von den \emph{Razor Pages}. Bei den \emph{Razor Pages} handelt es sich um eine Technologie von
Microsoft, die C\# Elemente im \emph{HTML-Markup} ermöglichen \cite{RazorPages}[vgl.].
Der Name \emph{Blazor} entstand aus der Kombination der zwei Wörter \emph{Browser} und
\emph{Razor}. Es wurde 2019 erstmalig von Microsoft
veröffentlicht, mit der Intention Webseiten oder auch \ac{spa} mithilfe von C\# zu entwickeln.
Dabei existieren zwei Varianten von Blazor:

\begin{itemize}
    \item Blazor Server
    \item Blazor WebAssembly
\end{itemize}
Der essenzielle Unterschied der beiden Varianten besteht darin, dass Blazor Server auf einem
Server gehostet wird und Blazor WebAssembly nativ im Browser läuft, dazu aber im späteren
Verlauf dieser Arbeit mehr \cite{WasIstBlazor}[vgl.].
\newline
\newline
Die Idee hinter \emph{Blazor} ist, die Codebasis sowohl im Frontend als auch im Backend mit
C\# abzubilden. Somit wird erreicht, dass langjährige C\#-Entwickler*innen mit ihrem vorhanden
Wissen als Fullstack-Entwickler*innen eingesetzt werden können.
