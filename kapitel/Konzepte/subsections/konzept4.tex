\section{Synthetische Daten erstellen}
\label{sec:Synthetische}
Das Konzept der \emph{Erstellung synthetischer Daten} markiert einen innovativen Ansatz innerhalb der Konzepte und weicht von den bisherigen Strategien ab. Dieser Ansatz verfolgt die Idee, ein Modell mit sämtlichen verfügbaren Daten zu trainieren und darauf aufbauend synthetische zukünftige Daten zu generieren. Ziel ist es, eine Art Simulation zu erstellen, die den Buchungsverlauf eines Hotels nachbildet.
\newline
\newline
Mittels dieser Simulation wird angestrebt, Vorhersagen darüber zu treffen, wie viele Buchungen für bestimmte Zimmerkategorien an bestimmten Tagen eingehen werden. Dadurch soll die Möglichkeit geschaffen werden, einen dynamischen Preis entsprechend dem erwarteten Buchungsverlauf zu gestalten.
\newline
\newline
Die Grundidee hinter dieser Vorgehensweise liegt in der Schaffung eines virtuellen Modells, welches basierend auf vergangenen Daten und Mustern potenzielle zukünftige Buchungen simuliert. Hierbei sollen verschiedene Szenarien durchgespielt werden, um die wahrscheinlichsten Buchungstrends abzuschätzen, und somit einen fundierten Ansatz für die dynamische Preisgestaltung zu generieren.