\section{Qt}
\label{sec:qt}
In der vorherigen Sektion wurde ein allgemeines Bild dargestellt, was ein Embedded System
ist und in welchen Varianten diese auftauchen. Weitergehend soll in dieser Sektion
vorgestellt werden, wie üblicherweise im \emph{Non-Deeply Embedded Systems} Bereich programmiert
wurde.
\newline
\newline
Ein großer Anteil eines \emph{Non-Deeply Embedded Systems} wird heutzutage durch die \ac{gui}
repräsentiert. Die \ac{gui} sollte intuitiv und zuverlässig sein.
Zudem ist es enorm wichtig, dass die \ac{gui} wenige Ressourcen verbraucht, schnell
reagiert und einfach einzubinden ist. Um diese Anforderung zu erfüllen, wird bis dato \emph{Qt} in
Kombination mit der Programmiersprache \emph{C++} verwendet \cite{QtOnEmbeddedLinux}[vgl.].

\subsection{Was ist Qt?}
\label{subsec:WasIstQt}
Qt ist ein Framework zum Erzeugen von GUIs auf mehreren Betriebssystemen. Es wurde 1990 von
\emph{Haarvard Nord} und \emph{Eirik Chambe-Eng} mit der Intention benutzerfreundliche
GUIs mit C++ zu entwickeln. Der Name \emph{Qt}
entstand dadurch, dass Haarvard den Buchstaben \emph{Q} in Emacs als sehr schön empfand. Zudem
steht das \emph{t} in Qt als
Abkürzung für das englische Wort \emph{Toolkit} \cite{qtStory}[vgl.].
\newline
\newline
Weitergehend sollte mit Qt die Möglichkeit geschaffen werden, mit nur einer Code-basis alle
Betriebssysteme abzudecken. Die Schwierigkeit bestand also darin, dasselbe Aussehen
und die gleiche Funktionalität über die verschiedenen Betriebssysteme zu
schaffen \cite{GettingStartedQt}[vgl.].
\newline
\newline
Qt ist in C++ entwickelt worden und verwendet zusätzlich noch einen Compiler welcher
\emph{\ac{moc}} genannt wird, mit welchem C++ um weitere Elemente erweitert wird. Der daraus
kompilierte Code folgt dem C++-Standard und ist somit mit jedem anderen C++ Compiler kompatibel.
Obwohl Qt ursprünglich dafür gedacht war, rein auf C++ zu basieren, wurden im Laufe der Zeit
mehrere Eweiterungen für Qt von der Community entwickelt. Somit kann Qt mit mehreren
Programmiersprachen benutzen werden, wie zum Beispiel Python oder Java.
\begin{figure}[h]
    \centering
    \includegraphics[width=0.4\textwidth, center]{StandDerTechnik/qtLogo}
    \caption[Qt Logo]{Qt Logo}
    \label{img:qtLogo}
\end{figure}
\subsection{Programmierbeispiel}
\label{subsec:programmierbeispiel}
Das Programmierbeispiel welches im Folgenden dargestellt wird, erzeugt ein Fenster mit dem Titel 
\emph{Meine erste Qt App}. Zudem ein Label welches \emph{Hello World} anzeigt und ein Button mit der
Aufschrift \emph{Exit}. Sobald der Button wird, schließt sich das Fenster.

\lstinputlisting[language=C++,caption={Qt Hello World Sourcecode},
    label=lst:qtHelloWorldSourceCode]{\srcloc/StandDerTechnik/qtHelloWorld.cpp}

Daraus ergibt sich das folgende Programm
\begin{figure}[h]
    \centering
    \includegraphics[width=0.5\textwidth, center]{StandDerTechnik/qtHelloWorldApp1}
    \caption[Qt Hello World App]{Qt Hello World App}
    \label{img:qtHelloWorldApp}
\end{figure}
\subsection{Widgets}
\label{subsec:widgets}
Zur Entwicklung einer \ac{gui} werden grafische Komponenten benötigt. Qt verwendet dafür sogenannte \emph{Widgets}. Widgets sind grafische
Komponenten um die Benutzeroberfläche zu gestalten. Ein Beispiel für eine solche Komponente wäre ein Button, welcher in der Sektion
\emph{\nameref{subsec:programmierbeispiel}} als \emph{btnExit} vorkam.
\newline
\newline
Die Widgets, die Qt zur Verfügung stellt sind in einer großen Klassenhierarchie zusammengesetzt
und diese Hierarchie könnte sich wie folgt dargestellt werden:
\newpage
\begin{figure}[h]
    \centering
    \includegraphics[width=\textwidth, center]{StandDerTechnik/qtWidgetClassHierachie}
    \caption[Qt Widgets Klassenhierachie]{Qt Widgets Klassenhierarchie
    \cite{GettingStartedQt}[vgl.]}
    \label{img:qtWidgetClassHierachie}
\end{figure}

Als Basisklasse ist ganz oben in der Klassenhierarchie die \emph{QObject} Klasse. Diese
enthält unter anderem den Signal- und Slot-Mechanismus. Auf diesen soll später noch genauer
eingegangen werden.
Weitergehend werden Widgets, die gemeinsame Funktionalitäten aufweisen zusammen gruppiert. Diese
Verhalten ist bei \emph{QPushButton} und \emph{QRadioButton} erkennbar. Beide Widgets sind
Buttons, die sich teilweise die gleichen Eigenschaften und Funktionen teilen
\cite{GettingStartedQt}[vgl.].

\subsection{Signal und Slot Konzept}
\label{subsec:signalslot}
Eine interaktive Benutzeroberfläche muss auf
Ereignisse reagieren können. Beispielsweise ist das Betätigen eines Buttons ein Auslöser für ein
Ereignis auf welches reagiert werden kann. Es gibt viele Methoden und Muster in welche ein
solches Ereignis-Aktionskonzept implementiert werden kann.

Ein \emph{Signal} ist im einfachen Sinne eine Nachricht, die versendet werden kann. Die Nachricht
signalisiert, dass sich der momentane Status eines Objektes geändert hat. Dahingegen ist ein
\emph{Slot} eine spezielle Funktion von einem Objekt, welche dann aufgerufen wird, wenn ein
bestimmtes \emph{Signal} gesendet wird \cite{GettingStartedQt}[vgl.].

Damit jeder \emph{Slot} auch weiß, auf welches \emph{Signal} reagiert werden soll, müssen diese verbunden werden. In der Sektion \emph{\nameref{subsec:programmierbeispiel}} war
folgende Zeile zu sehen:
\newpage
\begin{lstlisting}[language=C++, caption=Signal- und Slot-Beispiel, label=lst:SignalSlotBeispiel]
// Connect button with the App
QObject::connect(btnExit, SIGNAL(clicked()), &app, SLOT(quit()));

\end{lstlisting}

In dieser Codezeile fand die Verbindung zwischen einem Signal und einem Slot statt.
Wenn der Button betätigt wird, wird \emph{app} signalisiert und ruft \emph{quit()} auf. Die
Funktion \emph{quit()} beendet das Programm.

Einer der größten Vorteile dieser Methode im Gegensatz zu anderen Methoden ist, dass dadurch
n:m-Beziehungen abgebildet werden können. Das bedeutet also, dass sich ein Signal mit
beliebig viel Slots verbinden kann und dass sich ein Slot auf mehrere Signale verbinden kann
\cite{SignalSlotMechanismus}[vgl.].
\subsection{Raspberry Pi und Qt}
\label{subsec:rasPiUndQt}
Nachdem ein grober Überblick zu Qt geschaffen wurde, soll eine beispielhafte Anwendung
mithilfe von Qt auf dem Raspberry Pi implementiert werden. Bei dieser Anwendung soll beim
Betätigen eines Buttons, Daten von dem Raspberry Pi gelesen werden und auf der \ac{gui} angezeigt
werden. Dieses Beispiel ist von dem Embedded Systems 2 Labor inspiriert.

\subsubsection{RTIMULib}
\label{subsubsec:RTIMULib}
Die RTIMU Bibliothek war anfändlich dazu gedacht, mithilfe von C++ oder
Python Daten von einem Non-Deeply Embedded System zu lesen. Mittlerweile existiert diese
Bibliothek auch für andere Sprachen wie zum Beispiel C\#.
\newline
\newline
Mithilfe dieser Bibliothek soll für dieses Beispiel die Daten von dem Raspberry Pi gelesen
werden, um diese anschließend auf der \ac{gui} anzeigen zu können. Um RTIMU in einem QT
Projekt nutzen zu können, muss die Bibliothek zunächst in der \emph{.pro} Datei eingebunden
werden. Dies geschieht, indem die Zeile \emph{LIBS += -lRTIMULib} hinzugefügt wird.

\subsubsection{Hilfsklasse}
\label{subsubsec:Hilfsklasse}
Nachdem die Bibliothek hinzugefügt wurde, soll eine Hilfsklasse als Repräsentation der Daten
erzeugt werden. Das folgende Listing zeigt dies auf:

\begin{lstlisting}[language=C++, caption=RTIMU-Hilfsklasse, label=lst:Hilfsklasse]
class ReadData
{
private:
    float m_Temperature = 0.1f;
    float m_AirPressur = 0.1f;
    float m_Humidity = 0.1f;
    float m_xMagnetometer = 0.1f;
    float m_yMagnetometer = 0.1f;
    float m_zMagnetometer = 0.1f;

    RTIMUSettings* m_RTIMUSettings = nullptr;
    RTIMU* m_RTIMU = nullptr;
    RTPressure* m_RTPressure = nullptr;
    RTHumidity* m_RTHumidity = nullptr;

public:
    ReadData();
    void vRead(void);
};

\end{lstlisting}

Die Klasse enthält neben den im Listing \ref{lst:Hilfsklasse} gezeigten Code weitere
Methoden, um auf die privaten \emph{Member Variablen} zuzugreifen. Zudem müssen im Konstruktor
die Variablen konfiguriert und initialisiert werden. Dies geschieht folgendermaßen:

\begin{lstlisting}[language=C++, caption=RTIMU-Hilfsklasse-Konstruktor,
    label=lst:HilfsklasseKonstruktor]
ReadData::ReadData()
{
     // Define variables
     m_RTIMUSettings = new RTIMUSettings("RTIMULib");
     m_RTIMU = RTIMU::createIMU(pRTIMUSettings);
     m_RTPressure = RTPressure::createPressure(pRTIMUSettings);
     m_RTHumidity = RTHumidity::createHumidity(pRTIMUSettings);

     // Init
     pRTIMU->IMUInit();
     pRTIMU->setCompassEnable(true);
     pRTPressure->pressureInit();
     pRTHumidity->humidityInit();
}

\end{lstlisting}

Die letzte Komponente der Hilfsklasse, bildet die \emph{vRead} Methode ab, in der
schlussendlich die Daten gelesen werden:

\begin{lstlisting}[language=C++, caption=RTIMU-Hilfsklasse mit der Methode vRead,
    label=lst:vRead]
void ReadData::vRead(void)
{
    if (m_RTIMU->IMURead())
    {
        RTIMU_DATA RTIMUData = m_RTIMU->getIMUData();
        m_RTPressure->pressureRead(RTIMUData);
        m_RTHumidity->humidityRead(RTIMUData);

        m_AirPressur = RTIMUData.pressure;
        m_Temperature = RTIMUData.temperature;
        m_fHumidity = RTIMUData.humidity;

        RTIMUData.compass.normalize();
        m_xMagnetometer = RTIMUData.compass.x();
        m_yMagnetometer = RTIMUData.compass.y();
        m_zMagnetometer = RTIMUData.compass.z();
    }
}

\end{lstlisting}

\subsubsection{Benutzeroberfläche}
\label{subsubsec:QtGui}
Die Benutzeroberfläche in diesem Beispiel wurde schlicht gehalten. Sie besitzt insgesamt zwölf
QtLabels, wovon sechs dafür gedacht sind, um die Raspberry Pi Daten abzubilden. Weitergehend
enthält das Beispiel einen Button, der die Daten abrufen soll.
\newline
\newline
Der Button wurde mit einem Slot versehen der aufgerufen wird, wenn der Button betätigt wird. In
dem Slot wird die vRead Methode aufgerufen, um dann die Labels zu aktualisieren. Die
Implementierung des Slots sieht dann folgend aus:

\begin{lstlisting}[language=C++, caption=Slot Methode für den Update-Button,
    label=lst:updateSlot]
void MainWindow::on_pbUpdate_clicked()
{
    m_readData->vRead();

    double dValue = static_cast<double>(m_readData->getAirPressur());
    ui->lblLuftdruck->setText(QString::number(dValue));

    dValue = static_cast<double>(m_readData->getTemperature());
    ui->lblTemperatur->setText(QString::number(dValue));

    dValue = static_cast<double>(m_readData->getHumidity());
    ui->lblLuftfreutigkeit->setText(QString::number(dValue));

    dValue = static_cast<double>(m_readData->getMagnetometerX());
    ui->lblKompassX->setText(QString::number(dValue));

    dValue = static_cast<double>(m_readData->getMagnetometerY());
    ui->lblKompassY->setText(QString::number(dValue));

    dValue = static_cast<double>(m_readData->getMagnetometerZ());
    ui->lblKompassZ->setText(QString::number(dValue));
}

\end{lstlisting}

Das vollständige Programm ist in der folgenden Abbildung dargestellt:

\begin{figure}[h]
    \centering
    \includegraphics[width=0.55\textwidth, center]{StandDerTechnik/QtGUI}
    \caption[GUI der beispielhaften Anwendung]{GUI der beispielhaften Anwendung}
    \label{img:qtGui}
\end{figure}
