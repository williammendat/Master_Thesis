\subsection{Was ist Qt?}
\label{subsec:WasIstQt}
Qt ist ein Framework zum Erzeugen von GUIs auf mehreren Betriebssystemen. Es wurde 1990 von
\emph{Haarvard Nord} und \emph{Eirik Chambe-Eng} mit der Intention benutzerfreundliche
GUIs mit C++ zu entwickeln. Der Name \emph{Qt}
entstand dadurch, dass Haarvard den Buchstaben \emph{Q} in Emacs als sehr schön empfand. Zudem
steht das \emph{t} in Qt als
Abkürzung für das englische Wort \emph{Toolkit} \cite{qtStory}[vgl.].
\newline
\newline
Weitergehend sollte mit Qt die Möglichkeit geschaffen werden, mit nur einer Code-basis alle
Betriebssysteme abzudecken. Die Schwierigkeit bestand also darin, dasselbe Aussehen
und die gleiche Funktionalität über die verschiedenen Betriebssysteme zu
schaffen \cite{GettingStartedQt}[vgl.].
\newline
\newline
Qt ist in C++ entwickelt worden und verwendet zusätzlich noch einen Compiler welcher
\emph{\ac{moc}} genannt wird, mit welchem C++ um weitere Elemente erweitert wird. Der daraus
kompilierte Code folgt dem C++-Standard und ist somit mit jedem anderen C++ Compiler kompatibel.
Obwohl Qt ursprünglich dafür gedacht war, rein auf C++ zu basieren, wurden im Laufe der Zeit
mehrere Eweiterungen für Qt von der Community entwickelt. Somit kann Qt mit mehreren
Programmiersprachen benutzen werden, wie zum Beispiel Python oder Java.
\begin{figure}[h]
    \centering
    \includegraphics[width=0.4\textwidth, center]{StandDerTechnik/qtLogo}
    \caption[Qt Logo]{Qt Logo}
    \label{img:qtLogo}
\end{figure}